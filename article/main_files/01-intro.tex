%!TeX root=../injoit-rus.tex

\section{Введение}
\label{sec:intro}
Защита программного кода от несанкционированного анализа и модификации является одной из критически важных задач современной кибербезопасности. В условиях роста сложности программных систем и увеличения числа атак, направленных на кражу интеллектуальной собственности или эксплуатацию уязвимостей, традиционные методы защиты часто оказываются недостаточными. Реверс-инжиниринг представляет собой серьезную угрозу, позволяя злоумышленникам восстанавливать алгоритмы и логику работы программ \cite{Collberg97Survey}.

Одним из продвинутых методов противодействия реверс-инжинирингу является метаморфизм кода – техника, при которой программа способна изменять собственную структуру во время выполнения, сохраняя при этом исходную функциональность \cite{Szor05Metamorphic}. В отличие от полиморфизма, метаморфизм не полагается на шифрование, а использует набор семантически эквивалентных преобразований кода. Это приводит к генерации большого числа уникальных вариантов программы, что делает неэффективными сигнатурные методы обнаружения и значительно усложняет ручной анализ.

Несмотря на потенциальную эффективность метаморфизма, его практическое применение затруднено высокой сложностью и трудоемкостью ручной реализации техник обфускации. Существующие автоматизированные инструменты часто ограничены в своих возможностях или ориентированы на специфические платформы \cite{Schrittwieser16Survey}. В последние годы значительный прогресс в области искусственного интеллекта, в частности появление больших языковых моделей, обученных на огромных массивах кода \cite{Chen21Evaluating}, открывает новые перспективы для автоматизации сложных задач программной инженерии, включая метаморфизм кода.

В данной работе представлен инструментарий, разработанный для автоматизации процесса метаморфизма кода на языке Go с использованием LLM. Мы исследуем возможность применения современных нейронных сетей для генерации метаморфических преобразований, реализующих техники вставки "мертвого" кода и модификации потока управления. Целью работы является не только создание прототипа такого инструментария, но и количественная оценка его эффективности, включая анализ сохранения функциональной эквивалентности и степени достигаемой обфускации кода с использованием стандартных метрик сложности.

Статья организована следующим образом: Раздел 2 кратко рассматривает связанные работы. Раздел 3 описывает методологию и архитектуру разработанного инструментария. Раздел 4 представляет результаты экспериментальной оценки. В Разделе 5 приводятся заключение и обсуждаются направления будущих исследований.
