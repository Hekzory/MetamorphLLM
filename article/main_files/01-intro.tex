%!TeX root=../injoit-rus.tex

\section{Введение}
\label{sec:intro}
Защита программного кода от несанкционированного анализа и модификации стала критической задачей современной кибербезопасности. В эпоху усложнения программных систем и роста целенаправленных атак на интеллектуальную собственность традиционные методы защиты утрачивают эффективность. Реверс-инжиниринг представляет особую угрозу, позволяя злоумышленникам восстанавливать алгоритмы и внутреннюю логику программ \cite{Collberg97Survey}.

Одним из продвинутых методов противодействия реверс-инжинирингу является метаморфизм кода – техника, при которой программа способна изменять собственную структуру при сохранении исходной функциональности \cite{Szor05Metamorphic}. В отличие от полиморфизма, метаморфизм не полагается на шифрование, а использует набор семантически эквивалентных преобразований кода. Это приводит к генерации большого числа функционально идентичных, но структурно различных вариантов программы, что делает неэффективными сигнатурные методы обнаружения и значительно усложняет ручной анализ.

Несмотря на потенциальную эффективность метаморфизма, его практическое применение затруднено высокой сложностью и трудоемкостью ручной реализации техник обфускации. Существующие автоматизированные инструменты часто ограничены в своих возможностях или ориентированы на специфические платформы \cite{Schrittwieser16Survey}. В последние годы значительный прогресс в области искусственного интеллекта, в частности появление больших языковых моделей (LLM), обученных на огромных массивах кода \cite{Chen21Evaluating}, открывает новые перспективы для автоматизации сложных задач программной инженерии, включая метаморфизм кода.

В данной работе представлен инструментарий, разработанный для автоматизации процесса метаморфизма кода на языке Go с использованием LLM. Мы исследуем возможность применения современных нейронных сетей для генерации метаморфических преобразований, реализующих техники вставки \enquote{мертвого} кода и модификации потока управления. Целью работы является не только создание прототипа такого инструментария, но и количественная оценка его эффективности, включая анализ сохранения функциональной эквивалентности и степени достигаемой обфускации кода с использованием стандартных метрик сложности.

Статья организована следующим образом: Раздел 2 рассматривает связанные работы и существующие подходы к метаморфизму кода. Раздел 3 детально описывает методологию и архитектуру разработанного инструментария. Раздел 4 представляет систему метрик, используемых для объективной оценки эффективности метаморфических преобразований. В Разделе 5 приводятся результаты экспериментальной оценки и их всесторонний анализ. Завершает статью Раздел 6, содержащий заключение и обсуждение перспективных направлений будущих исследований.
