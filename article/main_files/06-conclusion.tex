\section{Заключение и Будущие направления}

В ходе выполнения данной работы был успешно разработан и апробирован инструментарий для автоматизации метаморфизма кода на языке Go, использующий возможности современных больших языковых моделей. Основной целью являлось исследование применимости нейронных сетей для генерации семантически эквивалентных, но структурно измененных вариантов кода, реализующих техники вставки "мертвого"\, кода и модификации потока управления.

Ключевым результатом работы является демонстрация принципиальной возможности автоматизации метаморфизма с помощью LLM. Созданный прототип инструментария с модульной архитектурой способен итеративно модифицировать Go-код, делегируя генерацию преобразований внешним API нейронных сетей. Проведенные эксперименты с тремя различными моделями LLM показали, что они способны генерировать код, который в большинстве случаев успешно компилируется и сохраняет исходную функциональность, особенно при использовании техники вставки "мертвого" кода.

Анализ метрик обфускации $LOC_{fin}$, $CC_{fin}$, $CogC_{fin}$ количественно подтвердил, что применение разработанного подхода приводит к значительному усложнению программного кода. Относительный прирост цикломатической и когнитивной сложности, свидетельствует о потенциале метода для затруднения анализа и понимания кода как автоматизированными средствами, так и человеком. Сравнительный анализ моделей выявил компромисс между уровнем достигаемой обфускации и стабильностью сохранения функциональности.

Вклад данной работы заключается в предложении нового подхода к автоматизации метаморфизма, основанного на использовании LLM и применимого к исходному коду Go. Проведенное сравнение различных LLM в контексте этой специфической задачи также представляет практический и научный интерес.

Несмотря на достигнутые результаты, выявленные ограничения определяют направления для дальнейших исследований. Основными вызовами остаются обеспечение стопроцентной гарантии сохранения функциональной эквивалентности при сложных преобразованиях потока управления, а также разработка более точных и комплексных метрик для оценки реальной стойкости обфусцированного кода к инструментам реверс-инжиниринга.

Перспективы дальнейших исследований включают:
\begin{itemize}
    \item Расширение набора реализуемых техник метаморфизма.
    \item Исследование и оптимизация методов взаимодействия с LLM, например prompt engineering, fine-tuning моделей на задачах преобразования кода.
    \item Интеграция механизмов обратной связи для адаптивного управления процессом метаморфизма.
    \item Разработка более совершенных метрик оценки стойкости обфускации.
    \item Тестирование инструментария на более крупных и сложных реальных проектах на Go для оценки практической применимости и масштабируемости.
\end{itemize}
Развитие предложенного подхода имеет потенциал для создания нового поколения интеллектуальных инструментов защиты программного обеспечения.