%!TeX root=../injoit-rus.tex

\section{Связанные работы}
\label{sec:literature}
Методы обфускации и метаморфизма кода являются предметом исследований в области компьютерной безопасности и программной инженерии на протяжении нескольких десятилетий. Фундаментальные техники обфускации, такие как вставка неиспользуемого кода, модификация потока управления, переименование идентификаторов и преобразование данных, были систематизированы и подробно описаны в основополагающих работах, например, Collberg и др. \cite{Collberg97Survey}. Метаморфизм, как развитие идей обфускации, фокусируется на самомодификации кода с целью уклонения от обнаружения, что детально рассмотрено в работе Szor \cite{Szor05Metamorphic}. Различные аспекты эволюции и реализации метаморфических техник также анализируются в обзорах Sharma и Sahay \cite{Sharma14Evolution} и Brezinski и Ferens \cite{Brezinski21Survey}.

Несмотря на глубокую теоретическую проработку, автоматизация метаморфизма остается сложной задачей. Существующие инструментарии часто либо являются коммерческими продуктами с закрытым исходным кодом, либо представляют собой академические прототипы или генераторы вредоносного ПО, ориентированные на низкоуровневые представления кода (байт-код, ассемблер) и специфические платформы \cite{Brezinski21Survey}. Обзоры средств защиты программного обеспечения, такие как работа Schrittwieser и др. \cite{Schrittwieser16Survey}, часто отмечают ограничения существующих общедоступных инструментов обфускации, которые редко реализуют сложные метаморфические преобразования и не обладают адаптивностью. Большинство подходов либо применяют предопределенный набор правил, либо требуют значительного ручного вмешательства для настройки.

Исследования в области применения машинного обучения и формальных методов чаще фокусировались на обнаружении и анализе метаморфического кода \cite{Wong06Hunting, Campion21Learning}, а не на его автоматизированной генерации с целью защиты легитимного ПО. Подходы к изучению правил трансформации из образцов \cite{Campion21Learning} интересны, но не решают задачу создания метаморфического кода по запросу.

Данная работа направлена на восполнение этого пробела. В отличие от классических инструментов, мы предлагаем подход, использующий возможности современных LLM для автоматизации метаморфических преобразований непосредственно на уровне исходного кода языка Go. Мы не фокусируемся на низкоуровневых представлениях и не привязываемся к конкретным архитектурам, делегируя интеллектуальную часть генерации преобразований внешним моделям ИИ, что отличает наш подход от большинства существующих решений.
